% This text is proprietary.
% It's a part of presentation made by myself.
% It may not used commercial.
% The noncommercial use such as private and study is free
% Sep. 2005 
% Author: Sascha Frank 
% University Freiburg 
% www.informatik.uni-freiburg.de/~frank/


\documentclass{beamer}
\usepackage[utf8]{inputenc}
\usepackage[T1]{fontenc}
\usepackage[spanish]{babel}
\usepackage{graphicx}
\usepackage{color}
\usepackage{listings}
%para poder usar begin comment y end document y asi comentar varias lineas
\usepackage{verbatim}

\begin{document}
\title{Proyecto Final\\Módulo computacional para análisis de videos tomados a través del método experimental LEED}
\author{SP2118}
\date{\today}

\frame{\titlepage}

\frame{\frametitle{Contenidos de la presentación}\tableofcontents}

\section{Intro...}

\frame{\frametitle{Intro...}
\begin{center}Descripción general\end{center}
}

\frame{\frametitle{Intro...}
\begin{center}¿Para quién es esto?\end{center}
}

\section{LEED (Low Energy Electron Diffraction)}

\frame{\frametitle{LEED (Low Energy Electron Diffraction)}
\begin{center}¿Qué es?/¿Cómo funciona?/Ley de Bragg\end{center}
}

\frame{\frametitle{LEED (Low Energy Electron Diffraction)}
\begin{center}¿Para qué?\end{center}
}

\frame{\frametitle{LEED (Low Energy Electron Diffraction)}
\begin{center}¿Quién lo hace en CR?/¿Y en el mundo?\end{center}
}

\section{OpenCV+Python}

\frame{\frametitle{OpenCV+Python}
\begin{center}Instalación...!!\end{center}
}

\frame{\frametitle{OpenCV+Python}
\begin{center}Fácil de usar\end{center}
}

\section{Funcionalidades básicas del programa}

\frame{\frametitle{Funcionalidades básicas del programa}
\begin{center}Transformador de videos\end{center}
}

\frame{\frametitle{Funcionalidades básicas del programa}
\begin{center}Detención y reanudación (q)/Interrupción (s)\end{center}
}

\frame{\frametitle{Funcionalidades básicas del programa}
\begin{center}Dibujado/Devolverse (sin borrar lo dibujado)\end{center}
}

\section{OpenCV... un poco más}

\frame{\frametitle{OpenCV... un poco más}
\begin{center}Lectura de videos/Escala de grises\end{center}
}

\frame{\frametitle{OpenCV... un poco más}
\begin{center}Dibujado sobre frames/Eventos\end{center}
}

\frame{\frametitle{OpenCV... un poco más}
\begin{center}Obtención de intensidades promedio/Máscaras\end{center}
}
\section{¿Qué hay extra?}

\frame{\frametitle{¿Qué hay extra?}
\begin{center}Bastante automatizado\end{center}
}

\frame{\frametitle{¿Qué hay extra?}
\begin{center}Se da la opción de respaldar en repo/Bastante ordenado/¿Qué se genera?\end{center}
}


\section{Conclusión y ¿qué falta?}
\frame{\frametitle{Conclusión y ¿qué falta?}
\begin{center}Conclusión\end{center}
}

\frame{\frametitle{Conclusión y ¿qué falta?}
\begin{center}¿Qué falta?/3D\end{center}
}

\frame{\frametitle{}
\begin{center}Gracias\\¿Preguntas y/o recomendaciones?\end{center}
}








\begin{comment}


\section{Determinación del desplazamiento y la velocidad a partir de la aceleración}
\frame{\frametitle{Determinación del desplazamiento y la velocidad a partir de la aceleración}
Segundo teorema fundamental del cálculo diferencial e integral:\\
\begin{itemize}
\item Dada una función $\displaystyle f(x)$ continua en el intervalo $\displaystyle [a,b]$ y sea $\displaystyle F(x)$ cualquier función primitiva de $\displaystyle f$, es decir, $\displaystyle F'(x)=f(x)$, entonces:\\
\begin{center}
$\displaystyle \int_{a}^{b}f(x)dx=F(b)-F(a)$
\end{center}
\end{itemize}
}


\frame{\frametitle{Determinación del desplazamiento y la velocidad a partir de la aceleración}
Ayudados del segundo teorema fundamental del cálculo:\\
\begin{center}
$\displaystyle v_{x}(t)=\frac{dx(t)}{dt}\Rightarrow v_{x}(t')dt'=\frac{dx(t')}{dt'}dt'\Rightarrow $ \\ $\displaystyle \int _{t_{0}}^{t} v_{x}(t')dt'=\int_{t_{0}}^{t} \frac{dx(t')}{dt'}dt'=x(t)-x(t_{0})$\\ $\displaystyle \Rightarrow x(t)=x_{0}+\int_{t_{0}}^{t}v_{x}(t')dt' $
\end{center}
}


\frame{\frametitle{Determinación del desplazamiento y la velocidad a partir de la aceleración}
E igualmente, para la aceleración:\\
\begin{center}
$\displaystyle a_{x}(t)=\frac{dv_{x}(t)}{dt}\Rightarrow$\\
$\displaystyle v_{x}(t)=v_{x0}+\int_{t_{0}}^{t}a_{x}(t')dt' $
\end{center}
}


\frame{\frametitle{Determinación del desplazamiento y la velocidad a partir de la aceleración}
Más adelante se trabajará con las demás componentes de los vectores de posición, velocidad y aceleración:\\
\begin{center}
$\displaystyle \vec{r}(t)=\vec{r}_{0}+\int_{t_{0}}^{t}\vec{v}(t')dt' $\\
$\displaystyle \vec{v}(t)=\vec{v}_{0}+\int_{t_{0}}^{t}\vec{a}(t')dt' $
\end{center}
}


\section{Movimiento con aceleración constante}
\frame{\frametitle{Movimiento con aceleración constante}
Se toma $\displaystyle a_{x}$ constante, y $\displaystyle t_{0}=0$:\\
\begin{center}
$\displaystyle v_{x}(t)=v_{x0}+\int_{t_{0}}^{t}a_{x}(t')dt' \Rightarrow v_{x}(t)=v_{x0}+a_{x}(t-0 ) $\\
$\displaystyle v_{x}(t)=v_{x0}+a_{x}t$
\end{center}
}

\frame{\frametitle{Movimiento con aceleración constante}
Se toma $\displaystyle a_{x}$ constante, y $\displaystyle t_{0}=0$:\\
\begin{center}
$\displaystyle v_{x}(t)=v_{x0}+a_{x}t \surd $\\
$\displaystyle x(t)=x_{0}+\int_{t_{0}}^{t}v_{x}(t')dt' \Rightarrow x(t)=x_{0}+\int_{t_{0}}^{t}(v_{x0}+a_{x}t')dt' \Rightarrow x(t)=x_{0}+v_{x0}t+\frac{1}{2}a_{x}t^{2}$
\end{center}
}

\frame{\frametitle{Movimiento con aceleración constante}
Se toma $\displaystyle a_{x}$ constante, y $\displaystyle t_{0}=0$:\\
\begin{center}
$\displaystyle v_{x}(t)=v_{x0}+a_{x}t \surd  $\\
$\displaystyle x(t)=x_{0}+v_{x0}t+\frac{1}{2}a_{x}t^{2} \surd $\\
$\displaystyle \bar{v}_{x}=\frac{x-x_{0}}{t}\Rightarrow  \bar{v}_{x}=\frac{1}{2}(v_{x0}+v_{x})$
\end{center}
}



\frame{\frametitle{Movimiento con aceleración constante}
Se toma $\displaystyle a_{x}$ constante, y $\displaystyle t_{0}=0$:\\
\begin{center}
$\displaystyle v_{x}(t)=v_{x0}+a_{x}t \surd $\\
$\displaystyle x(t)=x_{0}+v_{x0}t+\frac{1}{2}a_{x}t^{2} \surd $\\
$\displaystyle \bar{v}_{x}=\frac{1}{2}(v_{x0}+v_{x}) \surd $\\
$\displaystyle \bar{v}_{x}=\frac{x-x_{0}}{t}\Rightarrow x=x_{0}+\bar{v}_{x}t$
\end{center}
}


\frame{\frametitle{Movimiento con aceleración constante}
La última es un poco tediosa... truco:\\
\begin{center}
$\displaystyle v_{x}(t)=v_{x0}+a_{x}t\Rightarrow a_{x}t=v_{x}-v_{x0}$\\ \ \\
$\displaystyle x=x_{0}+v_{x0}t+\frac{1}{2}a_{x}t^{2} \Rightarrow (a_{x}t)^{2}=2a_{x}(x-x_{0})-2a_{x}v_{x0}t \Rightarrow $\\
$\displaystyle (v_{x}-v_{x0})^{2}=2a_{x}(x-x_{0})-2a_{x}v_{x0}t\Rightarrow $\\
$\displaystyle v_{x}^{2}+v_{x0}^{2}-2v_{x0}v_{x}=2a_{x}(x-x_{0})-2a_{x}v_{x0}t$\\ \ \\
$\displaystyle v_{x}^{2}=2v_{x0}v_{x}-2a_{x}v_{x0}t-v_{x0}^{2}+2a_{x}(x-x_{0})\Rightarrow $\\
$\displaystyle v_{x}^{2}=2v_{x0}(v_{x}-a_{x}t)-v_{x0}^{2}+2a_{x}(x-x_{0})\Rightarrow $\\
$\displaystyle v_{x}^{2}=v_{x0}^{2}+2a_{x}(x-x_{0}) $
\end{center}
}




\frame{\frametitle{Movimiento con aceleración constante}
Se toma $\displaystyle a_{x}$ constante, y $\displaystyle t_{0}=0$:\\
\begin{center}
$\displaystyle v_{x}(t)=v_{x0}+a_{x}t $\\
$\displaystyle x(t)=x_{0}+v_{x0}t+\frac{1}{2}a_{x}t^{2}$\\
$\displaystyle \bar{v}_{x}=\frac{1}{2}(v_{x0}+v_{x})$\\
$\displaystyle x=x_{0}+\bar{v}_{x}t$\\
$\displaystyle v_{x}^{2}=v_{x0}^{2}+2a_{x}(x-x_{0}) $
\end{center}
}


\section{Práctica - Parte 2}
\frame{\frametitle{Práctica - Parte 2}
\textit{Problema 2.50, Bauer}\\
¿Cuánto tiempo tarda un auto en acelerar desde el reposo hasta 22,2 m/s si la aceleración es constante y el auto avanza 243 m durante el período de aceleración?
}


\frame{\frametitle{Práctica - Parte 2}
\textit{Problema 2.50, Bauer}\\
\textit{Solución}\\
Es dado:\\
$\displaystyle v_{x0}=0 \ m/s, \ \ \  v_{xt_{f}}=22 \ m/s, \ \ \ \Delta x=x_{t_{f}}-x_{0}=243 \ m, \ \ \ a \ cte $
Se utiliza entonces:\\
$\displaystyle v_{x}^{2}=v_{x0}^{2}+2a_{x}(x-x_{0})\Rightarrow v_{xt_{f}}^{2}=2a_{x}(x_{t_{f}}-x_{0}) \Rightarrow a_{x}=\frac{v_{xt_{f}}^{2}}{2\Delta x}$\\
junto con:\\
$\displaystyle v_{x}=v_{x0}+a_{x}t\Rightarrow v_{xt_{f}}=a_{x}t_{f}\Rightarrow t_{f}=\frac{v_{x}t_{f}}{a_{x}}=\frac{2\Delta x}{v_{xt_{f}}}$

}


\frame{\frametitle{Práctica - Parte 2}
\textit{Problema 2.50, Bauer}\\
\textit{Solución}\\ \ \\
Evaluando:\\ \ \\
$\displaystyle t_{f}=\frac{2\Delta x}{v_{xt_{f}}}=\frac{2 \cdot 243 \ m}{22 \ m/s}=22 \ s \ $\\ \  \\
}


\frame{\frametitle{Práctica - Parte 2}
\textit{Problema 2.50, Bauer}\\
\textit{Solución}\\ \ \\
$\displaystyle t_{f}=22 \ s$\\ \ \\
Tiene sentido?\\
Usain Bolt $\rightarrow $ en 2007, 200 m en 19,75 s $\rightarrow $ aprox. 10,13 m/s
}



\frame{\frametitle{Práctica - Parte 2}
\textit{Problema 1, examen de suficiencia del II ciclo de 2011}.\\
(aclarar notación unidimensional \& no se puede utilizar a cte)\\
La aceleración de una partícula está dada por:
\begin{center}
$\displaystyle a=-2,0 \ \frac{m}{s^{2}}+\left( 3,0 \ \frac{m}{s^{3}} \right)t; \ \ \ v(0)=v_{0}; \ \ \ x(0)=0,0$
\end{center}
$\displaystyle x$ en metros y $\displaystyle t$ en segundos. \textbf{a}) Encuentre $\displaystyle v_{0}$ tal que $\displaystyle x(4)=x(0)$, \textbf{b}) Encuentre $\displaystyle v(4)$.\\
}


\frame{\frametitle{Práctica - Parte 2}
\textit{Problema 1, examen de suficiencia del II ciclo de 2011}.\\
\textit{Solución}\\
PARTE A:\\
Hay que hacer uso de la condición $\displaystyle x(4)=x(0)$; por lo tanto, se procede a encontrar $\displaystyle x(t)$:\\
$\displaystyle v(t)=v_{0}+\int_{0}^{t}a(t')dt'=v_{0}+\int_{0}^{t}(-2t'+3t')dt'=$\\
$\displaystyle v_{0}+\left( -2t+\frac{3t^{2}}{2} \right)\Rightarrow $\\
$\displaystyle x=x_{0}+\int_{0}^{t}v(t')dt'=0+\int_{0}^{t}\left(v_{0}+\left( -2t'+\frac{3t'^{2}}{2} \right) \right)dt'=$\\
$\displaystyle v_{0}t+\left( -t^{2}+\frac{t^{3}}{2} \right)$
}


\frame{\frametitle{Práctica - Parte 2}
\textit{Problema 1, examen de suficiencia del II ciclo de 2011}.\\
\textit{Solución}\\
$\displaystyle x(t)=v_{0}t+\left( -t^{2}+\frac{t^{3}}{2} \right)$\\
$\displaystyle x(4)=x(0)\Rightarrow 4\cdot v_{0}+(-4^{2}+4^{3}/2)=0\Rightarrow v_{0}=-4 \ m/s$\\
}


\frame{\frametitle{Práctica - Parte 2}
\textit{Problema 1, examen de suficiencia del II ciclo de 2011}.\\
\textit{Solución}\\
PARTE B:\\
$\displaystyle v(4)$=?\\ \ \\
$\displaystyle v(t)=-4 \ m/s + \left( -2t+\frac{3t^{2}}{2} \right)\Rightarrow v(4)=12 \ m/s$
}



\section{Caída libre}
\frame{\frametitle{Caída libre}
Movimiento completamente vertical ($\displaystyle \vec{a}=a_{y}\hat{y}=-g\hat{y}\Rightarrow a_{y}=-g$):
\begin{center}
$\displaystyle v_{y}(t)=v_{y0}-gt $\\
$\displaystyle y(t)=y_{0}+v_{y0}t-\frac{1}{2}gt^{2}$\\
$\displaystyle \bar{v}_{y}=\frac{1}{2}(v_{y0}+v_{y})$\\
$\displaystyle y=y_{0}+\bar{v}_{y}t$\\
$\displaystyle v_{y}^{2}=v_{y0}^{2}-2g(y-y_{0}) $
\end{center}
}


\frame{\frametitle{Caída libre}
\textit{Problema 2.88 del Bauer:}\\
Una chica está de pie al borde de un acantilado a 100 m arriba del suelo. Extiende el brazo y lanza una piedra directamente hacia arriba con una rapidez de 8 m/s.\\
a) ¿Cuánto tarda la piedra en tocar el suelo?\\
b) ¿Cuál es la velocidad de la piedra justo antes de tocar el suelo?
}

\frame{\frametitle{Caída libre}
\textit{Problema 2.88 del Bauer:}\\
\textit{Solución:}\\
Dado:\\
$\displaystyle v_{y0}=8 \ m/s$\\
$\displaystyle y_{0}=100 \ m$ \\ \ \\
Lo que se pide: $\displaystyle t$ tal que la piedra llega al fondo del acantilado, es decir, $\displaystyle y=0$
}



\frame{\frametitle{Caída libre}
\textit{Problema 2.88 del Bauer:}\\
\textit{Solución:}\\
Herramientas disponibles:\\
\begin{center}
$\displaystyle v_{y}(t)=v_{y0}-gt $\\
$\displaystyle y(t)=y_{0}+v_{y0}t-\frac{1}{2}gt^{2}$\\
$\displaystyle \bar{v}_{y}=\frac{1}{2}(v_{y0}+v_{y})$\\
$\displaystyle y=y_{0}+\bar{v}_{y}t$\\
$\displaystyle v_{y}^{2}=v_{y0}^{2}-2g(y-y_{0}) $
\end{center}
}

\frame{\frametitle{Caída libre}
\textit{Problema 2.88 del Bauer:}\\
\textit{Solución:}\\
Herramientas disponibles:\\
\begin{center}
$\displaystyle v_{y}(t)=v_{y0}-gt $\\
$\displaystyle y(t)=y_{0}+v_{y0}t-\frac{1}{2}gt^{2} \surd $\\
$\displaystyle \bar{v}_{y}=\frac{1}{2}(v_{y0}+v_{y})$\\
$\displaystyle y=y_{0}+\bar{v}_{y}t$\\
$\displaystyle v_{y}^{2}=v_{y0}^{2}-2g(y-y_{0}) $
\end{center}
}



\frame{\frametitle{Caída libre}
\textit{Problema 2.88 del Bauer:}\\
\textit{Solución:}\\
Selección y planteamiento:\\
\begin{center}
$\displaystyle y(t)=y_{0}+v_{y0}t-\frac{1}{2}gt^{2} \Rightarrow y(t)=100+8t-\frac{1}{2}9,81t^{2} $\\
$\displaystyle \Rightarrow y=0\rightarrow t=?$\\
$\displaystyle 0=100+8t-\frac{1}{2}9,81t^{2} \Rightarrow$\\
$\displaystyle t_{1}=-3,773 \ s$\\
$\displaystyle t_{2}=5,404 \ s$
\end{center}
}


\frame{\frametitle{Caída libre}
\textit{Problema 2.88 del Bauer:}\\
\textit{Solución:}\\
Despeje y solución:\\
\begin{center}
$\displaystyle y(t)=y_{0}+v_{y0}t-\frac{1}{2}gt^{2} \Rightarrow y(t)=100+8t-\frac{1}{2}9,81t^{2} $\\
$\displaystyle \Rightarrow y=0\rightarrow t=?$\\
$\displaystyle 0=100+8t-\frac{1}{2}9,81t^{2} \Rightarrow$\\
$\displaystyle t_{1}=-3,773 \ s$\\
$\displaystyle t_{2}=5,404 \ s \surd$
\end{center}
}


\section{Trabajo para la casa}
\frame{\frametitle{Trabajo para la casa}
\begin{itemize}
\item Investigue por qué en la foto que aparece en la figura 2.1 del Bauer el tren se ve así de borroso, a pesar de que una foto es instantánea.
\item Investigue lo que es la celeridad.
\item Ejemplo 2.5 del Bauer, página 53.
\item Problema 2.65 del Bauer, página 68.
\item Problema 2.67 del Bauer, página 68.
\item Problema 2.73 del Bauer, página 68.
\item Problema 2.96 del Bauer, página 68.
\item Problema 2.98 del Bauer, página 68.
\end{itemize}
}







\subsection{Subsection no.1.1  }
\frame{ 
Without title somethink is missing. 
}
\end{comment}

\begin{comment}
\section{Lo que vamos a estudiar... un adelanto visual}
\frame{\frametitle{Lo que vamos a estudiar... un adelanto visual}
\begin{center}
http://www.ndt-ed.org/EducationResources/HighSchool/Radiography/bremss \\ trahlung\_popup.html
\end{center}
}


\section{Sobre el teorema de Noether}
\frame{\frametitle{Sobre el teorema de Noether}
\begin{center}
$\displaystyle \phi (x)=\phi ' (x) + \alpha \Delta \phi (x)$ \ \ \ [1]
\end{center}
\begin{center}
$\displaystyle 0=\delta S=\int d^{4}x   \left(   \frac{\partial \mathcal{L}}{\partial \phi}\delta \phi + \frac{\partial \mathcal{L}}{\partial(\partial_{\mu} \phi)} \delta (\partial_{\mu} \phi)  \right)    = \int d^{4}x \left( \frac{\partial \mathcal{L}}{\partial \phi} \delta \phi -\partial_{\mu}\left(\frac{\partial \mathcal{L}}{\partial(\partial_{\mu}\phi)}\right)\delta \phi + \partial _{\mu}\left(\frac{\partial \mathcal{L}}{\partial(\partial_{\mu}\phi)} \delta \phi \right) \right)$ \ \ \ [2] \\
$\displaystyle S=\int L dt = \int \mathcal{L}(\phi, \partial_{\mu} \phi) d^{4}x$
\end{center}
}

\frame{\frametitle{Teorema de la divergencia del cálculo integral vectorial, múltiples dimensiones}
\begin{center}
$\displaystyle \int_{V} \frac{\partial F_{i}}{\partial x_{i}}dV = \oint_{S}F_{i}n_{i}dS$ \ \ \ [3]
\end{center}
\begin{center}
$\displaystyle \mathcal{L}(x)\rightarrow \mathcal{L}(x)+\alpha \partial_{\mu}\mathcal{J}^{\mu}(x)$ \ \ \ [4]
\end{center}
}

\frame{\frametitle{Pasos finales en el desarrollo de Noether}
\begin{center}
$\displaystyle \alpha \Delta \mathcal{L} = \frac{\partial \mathcal{L}}{\partial \phi}(\alpha \Delta \phi)+\frac{\partial \mathcal{L}}{\partial(\partial_{\mu}\phi)}(\alpha \Delta (\partial_{\mu}\phi))$\\
$\displaystyle =\frac{\partial \mathcal{L}}{\partial \phi}(\alpha \Delta \phi)+\frac{\partial \mathcal{L}}{\partial (\partial_{\mu}\phi)}\partial_{\mu}(\alpha \Delta \phi)$\\
$\displaystyle = \alpha \partial_{\mu}\left( \frac{\partial \mathcal{L}}{\partial(\partial_{\mu}\phi)}\Delta \phi \right)+\alpha \left( \frac{\partial \mathcal{L}}{\partial \phi} - \partial_{\mu} \left( \frac{\partial \mathcal{L}}{\partial (\partial_{\mu}\phi)} \right) \right) \Delta \phi$ \ \ \ [5]
\end{center}
}

\frame{\frametitle{Sobre la corriente conservada en el desarrollo de Noether}
\begin{center}
$\displaystyle \alpha \partial_{\mu}\mathcal{J}^{\mu}=\alpha \partial_{\mu} \left( \frac{\partial \mathcal{L}}{\partial (\partial_{\mu}\phi)} \Delta \phi \right) \Rightarrow \partial_{\mu}j^{\mu}=0, \ j^{\mu}(x)=\frac{\partial \mathcal{L}}{\partial(\partial_{\mu}\phi)}\Delta \phi - \mathcal{J}^{\mu} $ \ \ \ [6]
\end{center}
varios campos:
\begin{center}
$\displaystyle j^{\mu}(x) = \frac{\partial \mathcal{L}}{\partial (\partial_{\mu}\phi_{i})}\Delta \phi_{i} - \mathcal{J}^{\mu}$ \ \ \ [7]
\end{center}
}


\frame{\frametitle{Carga asociada a la corriente conservada}
\begin{center}
$\displaystyle Q=\int_{todo}j^{0}d^{3}x$ \ \ \ [8]
\end{center}
}




\section{El teorema de Noether y los diferentes Lagrangianos}
\frame{\frametitle{El teorema de Noether y los diferentes Lagrangianos}
\begin{itemize}
\item Klein-Gordon, Yukawa y QED.
\item Las corrientes del electromagnetismo normalmente se conservan.
\end{itemize}
}



\section{Ecuaciones de Maxwell, transformada de Fourier y la fórmula integral de Cauchy}
\frame{\frametitle{Ecuaciones de Maxwell, transformada de Fourier y la fórmula integral de Cauchy}
\begin{center}
$\displaystyle \partial_{\mu}F^{\mu \nu}=j^{\nu}$ \ \ \ [9]
\end{center}
\begin{center}
$\displaystyle F_{\mu \nu}=\partial_{\mu}A_{\nu}-\partial_{\nu}A_{\mu}$ \ \ \ [10]
\end{center}
\begin{center}
$\displaystyle \partial^{\mu}A_{\mu}=0$ \ \ \ [11]
\end{center}
}

\frame{\frametitle{Ecuaciones de Maxwell, transformada de Fourier y la fórmula integral de Cauchy}
\begin{center}
$\displaystyle f(x)= \int \frac{d^{4}k}{(2 \pi)^{4}}e^{-ik\cdot x} \tilde{f} (k)$ \ \ \ [13] \\
$\displaystyle \tilde{f}(k)= \int d^{4}x \ e^{ik\cdot x}f(x)$ \ \ \ [14]
\end{center}
}

\frame{\frametitle{Ecuaciones de Maxwell, transformada de Fourier y la fórmula integral de Cauchy}
\begin{center}
$\displaystyle f(z_{0})=\frac{1}{2 \pi i}\int_{C}\frac{f(z)dz}{z-z_{0}}$
\end{center}
}

\section{Cálculo clásico de Bremsstrahlung}
\frame{\frametitle{Cálculo clásico de Bremsstrahlung}
\begin{center}
$\displaystyle j^{\mu}=(1,\vec{0})^{\mu}\cdot e \delta ^{(3)}(\vec{x})=\int dt \ (1,\vec{0})^{\mu} \cdot e \delta^{(4)}(x-y(t)) \ \ t.q. \ \ y^{\mu}(t)=(t,\vec{0})^{\mu} $ \ \ \ [15]
\end{center}
generalizando:
\begin{center}
$\displaystyle j^{\mu}=e\int d\tau \frac{dy^{\mu}(\tau)}{d\tau}\delta ^{(4)}(x-y(\tau)) $ \ \ \ [16]
\end{center}
y en nuestro caso:
\begin{center}
$\displaystyle y^{\mu}(\tau)=\frac{p^{\mu}}{m}\tau \ \ t.q. \ \ \tau <0$\\
$\displaystyle y^{\mu}(\tau)=\frac{p'^{\mu}}{m}\tau \ \ t.q. \ \ \tau >0$ \ \ \ [17]
\end{center}
}

\frame{\frametitle{Cálculo clásico de Bremsstrahlung}
Combinando todo lo anterior:
\begin{center}
$\displaystyle j^{\mu}(x)=e\int_{0}^{\infty}d\tau \frac{p'^{\mu}}{m}\delta^{(4)}\left( x-\frac{p'}{m}\tau \right)+e\int_{-\infty}^{0}d\tau \frac{p^{\mu}}{m}\delta^{(4)}\left( x-\frac{p}{m}\tau \right) $ \ \ \ [18]
\end{center}
}


\frame{\frametitle{Cálculo clásico de Bremsstrahlung}
Aplicando transformada de Fourier:
\begin{center}
$\displaystyle \tilde{j}^{\mu}(k)=\int d^{4}x \ e^{ik\cdot x}j^{\mu}(x) =e\int_{0}^{\infty}d\tau \frac{p'^{\mu}}{m}e^{i(k\cdot p'/m+i\epsilon)\tau}+e\int_{-\infty}^{0}d\tau \ \frac{p^{\mu}}{m}e^{i(k\cdot p/m-i\epsilon)\tau}$\\
$\displaystyle = ie\left( \frac{p'^{\mu}}{k\cdot p'+i\epsilon}-\frac{p^{\mu}}{k\cdot p-i\epsilon} \right)$ \ \ \ [19]
\end{center}
}


\frame{\frametitle{Cálculo clásico de Bremsstrahlung}
Haciendo uso del gauge de Lorentz:
\begin{center}
$\displaystyle \partial^{2}A^{\mu}=j^{\mu} \Rightarrow \int d^{4}x \ e^{ik\cdot x} \ \partial^{2}A^{\mu}=\int d^{4}x \ e^{ik\cdot x} j^{\mu} $ \ \ \ [20]
\end{center}
con lo que se obtiene:
\begin{center}
$\displaystyle \tilde{A}^{\mu}(k)=-\frac{1}{k^{2}}\tilde{j}^{\mu}(k) $ \ \ \ [21]
\end{center}
}

\frame{\frametitle{Cálculo clásico de Bremsstrahlung}
Introduciendo la corriente:
\begin{center}
$\displaystyle \tilde{A}^{\mu}(k)=\frac{-ie}{k^{2}}\left( \frac{p'^{\mu}}{k\cdot p'+i\epsilon}-\frac{p^{\mu}}{k\cdot p-i\epsilon} \right)$ \ \ \ [22]
\end{center}
y aplicando transformada inversa:
\begin{center}
$\displaystyle A^{\mu}(x)=\int\frac{d^{4}x}{(2 \pi)^{4}}e^{-ik\cdot x}\frac{-ie}{k^{2}}\left( \frac{p'^{\mu}}{k\cdot p'+i\epsilon}-\frac{p^{\mu}}{k\cdot p-i\epsilon} \right) $ \ \ \ [23]
\end{center}
}


\frame{\frametitle{Cálculo clásico de Bremsstrahlung}
Polo superior:
\begin{center}
$\displaystyle A^{\mu}(x)=\int\frac{d^{3}x}{(2 \pi)^{3}}e^{i\vec{k}\cdot \vec{x}}e^{-i(\vec{k}\cdot \vec{p}/p^{0})t}\frac{(2 \pi i)(ie)}{(2 \pi)k^{2}}\frac{p^{\mu}}{p^{0}} $ \ \ \ [24]
\end{center}
y evaluando en $\displaystyle p^{\mu}=(p^{0},\vec{0})^{\mu}$:
\begin{center}
$\displaystyle A^{\mu}(x)=\int \frac{d^{3}x}{(2 \pi)^{3}}e^{i\vec{k}\cdot \vec{x}}\frac{e}{\vec{k}^{2}}\cdot (1,\vec{0})^{\mu}$ \ \ \ [25]
\end{center}
}

\frame{\frametitle{Cálculo clásico de Bremsstrahlung}
Polo inferior análogo al superior:
\begin{center}
$\displaystyle A^{\mu}(x)=-\int\frac{d^{3}x}{(2 \pi)^{3}}e^{i\vec{k}\cdot \vec{x}}e^{-i(\vec{k}\cdot \vec{p'}/p^{0})t}\frac{e}{\left( \frac{\vec{k}\cdot \vec{p'}}{p'^{0}}-\vec{k} \right)^{2}}\frac{p'^{\mu}}{p'^{0}} $ \ \ \ [26]
\end{center}
}

\frame{\frametitle{Cálculo clásico de Bremsstrahlung}
Polos restantes:
\begin{center}
$\displaystyle A^{\mu}_{rad}(x)= \left. \int \frac{d^{3}k}{(2 \pi)^{3}}\frac{-e}{2\mid \vec{k}\mid} \left \{ e^{-ik\cdot x}\left( \frac{p'^{\mu}}{k\cdot p'}-\frac{p^{\mu}}{k\cdot p} \right) + c.c. \right \} \right|_{k^{0}=\mid \vec{k} \mid}$\\
$\displaystyle = \left. Re \int \frac{d^{3}k}{(2 \pi)^{3}}\mathcal{A}^{\mu}(\vec{k})e^{-ik\cdot x} \right|_{k^{0}=\mid \vec{k} \mid} $ \ \ \ [27]
\end{center}
con:
\begin{center}
$\displaystyle \mathcal{A}^{\mu}(\vec{k})= \left. \frac{-e}{\mid \vec{k} \mid}\left( \frac{p'^{\mu}}{k\cdot p'}-\frac{p^{\mu}}{k\cdot p} \right) \right|_{k^{0}=\mid \vec{k} \mid} $ \ \ \ [28]
\end{center}
}


\frame{\frametitle{Cálculo clásico de Bremsstrahlung}
Energía de la radiación:
\begin{center}
$\displaystyle e_{rad}=\left.\int \frac{d^{3}k}{(2 \pi )^{3}}\frac{e^{2}}{2}\left( \frac{2p\cdot p'}{(k\cdot p')(k\cdot p)}-\frac{m^{2}}{(k\cdot p')^{2}}-\frac{m^{2}}{(k\cdot p)^{2}} \right) \right  |_{k^{0}=\mid \vec{k} \mid}$ \ \ \ [29]
\end{center}
}



\section{Tomando un marco de referencia en específico}
\frame{\frametitle{Tomando un marco de referencia en específico}
Se elige un marco en el que $\displaystyle p^{0}=p'^{0}=E$, tal que:
\begin{center}
$\displaystyle k^{\mu}=(k,\vec{k}) \ \ \ \ p^{\mu}=E(1,\vec{v}) \ \ \ \ p'^{\mu}=E(1,\vec{v'})$ \ \ \ [30] \\
\end{center}
\begin{center}
\tiny
$\displaystyle e_{rad}=\left.\int \frac{\mid \vec{k} \mid ^{2} dk \ d\Omega _{\hat{k}}}{4\pi (2 \pi )^{2}}e^{2}\left( \frac{2E^{2}(1-\vec{v}\cdot \vec{v'})}{\mid \vec{k} \mid ^{2} E^{2} (1-\hat{k}\cdot \vec{v})(1-\hat{k}\cdot \vec{v'})}- \frac{m^{2}}{\mid \vec{k} \mid ^{2} E^{2}(1-\hat{k}\cdot \vec{v'})^{2}}-\frac{m^{2}}{\mid \vec{k} \mid ^{2} E^{2}(1-\hat{k}\cdot \vec{v})^{2}} \right) \right  |_{k^{0}=\mid \vec{k} \mid}$ \ \ \ [31]
\end{center}
}

\frame{\frametitle{Tomando un marco de referencia en específico}
\begin{center}
$\displaystyle e_{rad}=\int_{0}^{\infty} dk \ \mathcal{I}(\vec{v},\vec{v'})$ \ \ \ [32]
\end{center}
\begin{center}
$\displaystyle e_{rad}=\int_{0}^{k_{max}} dk \ \mathcal{I}(\vec{v},\vec{v'})=k_{max}\mathcal{I}(\vec{v},\vec{v'}) $ \ \ \ [32]
\end{center}
}


\section{Para qué sirve todo esto? Y una posible fuente de error...}
\frame{\frametitle{}
\begin{center}
\begin{itemize}
\item Una posible fuente de error...\\
\item Y para qué sirve todo esto?
\end{itemize}
\end{center}
}


\frame{\frametitle{}
\begin{center}
Gracias
\end{center}
}













\section{Section no. 2} 
\subsection{Lists I}
\frame{\frametitle{unnumbered lists}
\begin{itemize}
\item Introduction to  \LaTeX  
\item Course 2 
\item Termpapers and presentations with \LaTeX 
\item Beamer class
\end{itemize} 
}

\frame{\frametitle{lists with pause}
\begin{itemize}
\item Introduction to  \LaTeX \pause 
\item Course 2 \pause 
\item Termpapers and presentations with \LaTeX \pause 
\item Beamer class
\end{itemize} 
}

\subsection{Lists II}
\frame{\frametitle{numbered lists}
\begin{enumerate}
\item Introduction to  \LaTeX  
\item Course 2 
\item Termpapers and presentations with \LaTeX 
\item Beamer class
\end{enumerate}
}
\frame{\frametitle{numbered lists with pause}
\begin{enumerate}
\item Introduction to  \LaTeX \pause 
\item Course 2 \pause 
\item Termpapers and presentations with \LaTeX \pause 
\item Beamer class
\end{enumerate}
}

\section{Section no.3} 
\subsection{Tables}
\frame{\frametitle{Tables}
\begin{tabular}{|c|c|c|}
\hline
\textbf{Date} & \textbf{Instructor} & \textbf{Title} \\
\hline
WS 04/05 & Sascha Frank & First steps with  \LaTeX  \\
\hline
SS 05 & Sascha Frank & \LaTeX \ Course serial \\
\hline
\end{tabular}}


\frame{\frametitle{Tables with pause}
\begin{tabular}{c c c}
A & B & C \\ 
\pause 
1 & 2 & 3 \\  
\pause 
A & B & C \\ 
\end{tabular} }


\section{Section no. 4}
\subsection{blocs}
\frame{\frametitle{blocs}

\begin{block}{title of the bloc}
bloc text
\end{block}

\begin{exampleblock}{title of the bloc}
bloc text
\end{exampleblock}


\begin{alertblock}{title of the bloc}
bloc text
\end{alertblock}
}

\end{comment}

\end{document}

